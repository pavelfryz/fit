\section{Implementace protokolu Otway Rees}
V~tomto příkladě si vyzkoušíte implementaci a~verifikaci \emph{Otway Rees}
protokolu, který používá symetrickou kryptografii.

\begin{enumerate}
  %%%>==== 1.krok
  \item Po spuštění nástroje zadávajte kód protokolu v~hlavní editovací části okna.

  %%%>==== 2.krok
  \item Na~úvod deklarujeme funkci pro symetrický klíč. Protože se jedná o~tajný
  klíč označíme funkci klíčovým slovem \texttt{secret}. Dále deklarujeme nový typ
  pro distribuovaný klíč \texttt{Key}.
  \begin{lstlisting}[name=OtwayRees]
secret k: Function;
usertype Key;
  \end{lstlisting}

  %%%>==== 3.krok
  \item Dále deklarujeme protokol klíčovým slovom \texttt{protocol} nasledovaný jeho jménem a~jmény účastníků.
  \begin{lstlisting}[name=OtwayRees]
protocol OtwayRees(A,B,S)
{
  \end{lstlisting}

  %%%>==== 4.krok
  \item Následuje definice jednotlivých uživatelů. Definice účastníka začíná klíčovým slovem \texttt{role}.
  Prvně definujeme účastníka \texttt{A} a~deklarujeme lokální proměnné pro nonce \texttt{Na} a~\texttt{M},
  a~pro obdržený klíč \texttt{Kab}.
  \begin{lstlisting}[name=OtwayRees]
  role A
  {
    const Na: Nonce;
    const M: Nonce;
    var Kab: Key;
  \end{lstlisting}

  %%%>==== 5.krok
  \item Dále definujem příchozí a~odchozí zprávy účastníka \texttt{A}.
  Příchozí a~odchozí zpráva začíná klíčovým slovem \texttt{recv}, respektive \texttt{send} následované označením zprávy.
  V~kulatých závorkách poté následuje od koho zpráva pochází, příjemce zprávy a~text zprávy.
  \begin{lstlisting}[name=OtwayRees]
    send_1(A,B,(M,A,B,{Na,M,A,B}k(A,S)));
    recv_4(B,A,(M,{Na,Kab}pk(A,S)));
  \end{lstlisting}
  kde \texttt{\{x\}pk(A,S)} značí zprávu \texttt{x} zašifrovanou sdíleným klíčem \texttt{A} a~\texttt{S}.

  %%%>==== 6.krok
  \item Nakonec přidáme bezpečnostní požadavky protokolu. V~našem případě se jedná o~utajení distribuovaného klíče \texttt{Kab}
  a~neinjektivní synchronizaci.
  \begin{lstlisting}[name=OtwayRees]
    claim_A1(A,Secret,Kab);
    claim_A2(A,Nisynch);
  }
  \end{lstlisting}

  %%%>==== 7.krok
  \item Obdobně přidáme i~účastníka \texttt{B}. Protože účastníci musí být schopni
  dešifrovat příchozí zprávy, musíme u~účasníka \texttt{B} navíc
  deklarovat proměnné \texttt{T1} a~\texttt{T2} typu \texttt{Ticket}, do kterých
  můžeme uložit zašifrovanou zprávu.
  \begin{lstlisting}[name=OtwayRees]
  role B
  {
    const Nb: Nonce;
    var M: Nonce;
    var Kab: Key;
    var T1, T2: Ticket;
    recv_1(A,B,(M,A,B,T1));
    send_2(B,S,(M,A,B,T1,{Nb,M,A,B}k(B,S)));
    recv_3(S,B,(M,T2,{Nb,Kab}k(B,S)));
    send_4(B,A,(M,T2));
    claim_B1(B,Secret,Kab);
    claim_B2(B,Nisynch);
  }
  \end{lstlisting}

  %%%>==== 8.krok
  \item Dále přidáme server \texttt{S}.
  \begin{lstlisting}[name=OtwayRees]
  role S
  {
    const Kab: Key;
    var Na, Nb, M: Nonce;
    recv_2(B,S,(M,A,B,{Na,M,A,B}k(A,S),{Nb,M,A,B}k(B,S)));
    send_3(S,B,(M,{Na,Kab}k(A,S),{Nb,Kab}k(B,S)));
  }
}
  \end{lstlisting}

  %%%>==== 9.krok
  \item Nakonec přidáme nedůvěryhodného klienta, jehož sdílený klíč byl zpronevěřen útočníkem.
  \begin{lstlisting}[name=OtwayRees]
const C,D: Agent;
untrusted C;
compromised k(C,D);
compromised k(D,C);
  \end{lstlisting}

  %%%>==== 10.krok
  \item Spuťte verifikaci protokolu(v menu \texttt{Verify->Verify protocol}), a~projděte výsledky verifikace. Nechte
  si zobrazit případný útok.
\end{enumerate}