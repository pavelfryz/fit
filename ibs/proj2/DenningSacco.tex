\section{Implementace protokolu Denning-Sacco}
V~tomto příkladě si vyzkoušíte implementaci a~verifikaci protokolu \emph{Denning-Sacco}.

\begin{enumerate}
  %%%>==== 1.krok
  \item Po spuštění nástroje zadávajte kód protokolu v~hlavní editovací části okna.

  %%%>==== 2.krok
  \item Na~úvod deklarujeme funkci pro symetrický klíč. Protože se jedná o~tajný
  klíč označíme funkci klíčovým slovem \texttt{secret}. Dále deklarujeme nový typ
  pro distribuovaný klíč \texttt{Key} a~typ \texttt{Timestamp} pro časovou značku.
  \begin{lstlisting}[name=DenningSacco]
secret k: Function;
usertype Key;
usertype Timestamp;
  \end{lstlisting}

  %%%>==== 3.krok
  \item Dále deklarujeme protokol klíčovým slovom \texttt{protocol} nasledovaný jeho jménem a~jmény účastníků.
  \begin{lstlisting}[name=DenningSacco]
protocol DenningSacco(A,B,S)
{
  \end{lstlisting}

  %%%>==== 4.krok
  \item Následuje definice jednotlivých uživatelů. Definice účastníka začíná klíčovým slovem \texttt{role}.
  Prvně definujeme účastníka \texttt{A} a~deklarujeme lokální proměnné pro časovou známku \texttt{T},
  pro obdržený klíč \texttt{Kab} a~ticket pro přeposílanou zprávu \texttt{M}.
  \begin{lstlisting}[name=DenningSacco]
  role A
  {
    var T: Timestamp;
    var M: Ticket;
    var Kab: Key;
  \end{lstlisting}

  %%%>==== 5.krok
  \item Dále definujem příchozí a~odchozí zprávy účastníka \texttt{A}.
  Příchozí a~odchozí zpráva začíná klíčovým slovem \texttt{recv}, respektive \texttt{send} následované označením zprávy.
  V~kulatých závorkách poté následuje od koho zpráva pochází, příjemce zprávy a~text zprávy.
  \begin{lstlisting}[name=DenningSacco]
    send_1(A,S,(A,B));
    recv_2(S,A,{B,Kab,T,M}k(A,S));
    send_3(A,B,M);
  \end{lstlisting}
  kde \texttt{\{x\}pk(A,S)} značí zprávu \texttt{x} zašifrovanou sdíleným klíčem \texttt{A} a~\texttt{S}.

  %%%>==== 6.krok
  \item Nakonec přidáme bezpečnostní požadavky protokolu. V~našem případě se jedná o~utajení distribuovaného klíče \texttt{Kab}
  a~neinjektivní synchronizaci a~agreement.
  \begin{lstlisting}[name=DenningSacco]
    claim_A1(A,Secret,Kab);
    claim_A2(A,Nisynch);
    claim_A3(A,Niagree);
  }
  \end{lstlisting}

  %%%>==== 7.krok
  \item Obdobně přidáme i~účastníka \texttt{B}.
  \begin{lstlisting}[name=DenningSacco]
  role B
  {
    var T: Timestamp;
    var Kab: Key;
    recv_3(A,B,{Kab,A,T}k(B,S));
    claim_B1(B,Secret,Kab);
    claim_B2(B,Nisynch);
    claim_B3(B,Niagree);
  }
  \end{lstlisting}

  %%%>==== 8.krok
  \item Dále přidáme server \texttt{S}.
  \begin{lstlisting}[name=DenningSacco]
  role S
  {
    const Kab: Key;
    const T: Timestamp;
    recv_1(A,S,(A,B));
    send_2(S,A,{B,Kab,T,{Kab,A,T}k(B,S)}k(A,S));
  }
}
  \end{lstlisting}

  %%%>==== 9.krok
  \item Nakonec přidáme nedůvěryhodného klienta, jehož sdílený klíč byl zpronevěřen útočníkem.
  \begin{lstlisting}[name=DenningSacco]
const C,D: Agent;
untrusted C;
compromised k(C,D);
compromised k(D,C);
  \end{lstlisting}

  %%%>==== 10.krok
  \item Spuťte verifikaci protokolu(v menu \texttt{Verify->Verify protocol}), a~projděte výsledky verifikace. Nechte
  si zobrazit případný útok.
\end{enumerate}