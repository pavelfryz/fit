%==============================================================================
% Název: IBS - Laboratorní cvičení
% Datum: 4. 5. 2012
% Autor: Pavel Frýz
%==============================================================================
% kódování: utf-8
% překlad: make
%==============================================================================

\documentclass[a4paper,final]{article}
\usepackage{amsmath,amssymb}
\usepackage[czech]{babel}
\usepackage[utf8]{inputenc}
\usepackage[T1]{fontenc}
\usepackage{url}
\usepackage{thumbpdf}
\usepackage{listings}
\usepackage{ifthen}
\usepackage[top=3cm, left=3cm, text={15cm, 24cm}, ignorefoot]{geometry}
\DeclareUrlCommand\url{\def\UrlLeft{<}\def\UrlRight{>} \urlstyle{tt}}
\renewcommand\uv[1]{\quotedblbase #1\textquotedblleft}%

\setlength{\parindent}{0em}

\ifx\pdfoutput\undefined  % nejedeme pod pdftexem
  \usepackage{graphics}
  \usepackage{epsfig}
\else % je to pdftex !
  \usepackage[pdftex]{graphicx}
  \usepackage{color}
  \usepackage[unicode,colorlinks,linkcolor=black,citecolor=black,urlcolor=black,hyperindex,plainpages=false,pdftex]{hyperref}
  \definecolor{links}{gray}{0}
  \definecolor{anchors}{gray}{0}
  \def\AnchorColor{anchors}
  \def\LinkColor{links}
  \def\pdfBorderAttrs{/Border [0 0 0] }
  \pdfcompresslevel=9
\fi

%------------------------------------------------------------------------------
\title{\bfseries{IBS - Laboratorní cvičení}}
\date{\today}
\author{Pavel Frýz}
%------------------------------------------------------------------------------

\begin{document}

\maketitle      % Titulok

\section*{Cíl úlohy}
\begin{itemize}
  \item Seznámit se s~nástrojom Scyther.
  \item Vyzkoušet si implementaci symetrických a~asymetrických bezpečnostních protokolů.
  \item Vyzkoušet si verifikaci bezpečnostních protokolů v~nástroji Scyther.
\end{itemize}

\section*{Studijní materiály}
\begin{itemize}
  \item V~této úloze bude cílem implementovat protokoly
  \emph{Needham-Schroeder Public Key}, \emph{Otway Rees} a~protokol \emph{Denning-Sacco}.
  \item Studijní materiály
    \begin{itemize}
      \item Needham-Schroeder Public Key\\
      \url{http://www.lsv.ens-cachan.fr/Software/spore/nspk.html}\\
      Roger Needham and Michael Schroeder. Using encryption for authentification
      in large networks of computers. \emph{Communications of the ACM}, 21(12), December
      1978.
      \item Otway Rees\\
      \url{http://www.lsv.ens-cachan.fr/Software/spore/otwayRees.html}\\
      John Clark and Jeremy Jacob. A~survey of authentication protocol literature : Version 1.0., November 1997.
      \item Denning-Sacco\\
      \url{http://www.lsv.ens-cachan.fr/Software/spore/denningSacco.html}\\
      D.\,Denning and G.\,Sacco. Timestamps in key distributed protocols.
      \emph{Communication of the ACM}, 24(8):533--535, 1981.
    \end{itemize}
\end{itemize}

\section*{Příprava prostředí}
\begin{itemize}
  \item Stáhněte a~nainstalujte knihovnu GraphViz-Graph Visualization Software:\\
  \url{http://www.graphviz.org/Download.php}.
  \item Stáhněte instalační soubor Pythonu 2.x a~spusťte instalaci:\\
  \url{http://www.python.org/download/}
  \item Stáhněte a~nainstalujte knihovnu wxPython, vyberte verzi odpovídající verzi Pythonu:\\
  \url{http://www.wxpython.org/download.php}
  \item Stáhněte a~rozbalte archiv s~nástrojem Scyther:\\
  \url{http://people.inf.ethz.ch/cremersc/scyther/install-generic.html}
  \item Spusťte soubor \texttt{scyther-gui.py} z~adresáře aplikace.
\end{itemize}

\lstset{numbers=left, stepnumber=1, numberstyle=\tiny,
        tabsize=1, xleftmargin=0.6cm, basicstyle=\ttfamily,
        morecomment=[l]\#}
%------------------------------------------------------------------------------
\section{Implementace protokolu Needham-Schroeder Public Key}
V~tomto příkladě si vyzkoušíte implementaci a~verifikaci \emph{Needham-Schroeder Public Key}
protokolu, který slouží pro obousměrnou autentizaci pomocí důvěryhodného serveru.

\begin{enumerate}
  %%%>==== 1.krok
  \item Po spuštění nástroje zadávajte kód protokolu v~hlavní editovací části okna.

  %%%>==== 2.krok
  \item Na~úvod deklarujeme funkce pro veřejné a~soukromé klíče. Deklarace pomocí klíčového slova \texttt{const},
  učiní funkci \texttt{pk} veřejnou. Pomocí slova \texttt{secret} označíme pak funkci \texttt{sk} jako tajnou.
  \begin{lstlisting}[name=NSPK]
const pk: Function;
secret sk: Function;
  \end{lstlisting}

  %%%>==== 3.krok
  \item Následně deklarujeme, že tyto funkce tvoří pár asymetrických klíčů.
  \begin{lstlisting}[name=NSPK]
inversekeys(pk,sk);
  \end{lstlisting}

  %%%>==== 4.krok
  \item Dále deklarujeme protokol klíčovým slovom \texttt{protocol} nasledovaný jeho jménem a~jmény účastníků.
  \begin{lstlisting}[name=NSPK]
protocol NeedhamSchroeder(A,B,S)
{
  \end{lstlisting}

  %%%>==== 5.krok
  \item Následuje definice jednotlivých uživatelů. Definice účastníka začíná klíčovým slovem \texttt{role}.
  Prvně definujeme účastníka \texttt{A} a~deklarujeme lokální proměnné pro nonce \texttt{Na}, který
  vytváří, a~pro nonce \texttt{Nb}, který obdrží od účastníka \texttt{B}.
  \begin{lstlisting}[name=NSPK]
  role A
  {
    const Na: Nonce;
    var Nb: Nonce;
  \end{lstlisting}

  %%%>==== 6.krok
  \item Dále definujem příchozí a~odchozí zprávy účastníka \texttt{A}.
  Příchozí a~odchozí zpráva začíná klíčovým slovem \texttt{recv}, respektive \texttt{send} následované označením zprávy.
  V~kulatých závorkách poté následuje od koho zpráva pochází, příjemce zprávy a~text zprávy.
  \begin{lstlisting}[name=NSPK]
    send_1(A,S,(A,B));
    recv_2(S,A,{pk(B),B}sk(S));
    send_3(A,B,{Na,A}pk(B));
    recv_6(B,A,{Na,Nb}pk(A));
    send_7(A,B,{Nb}pk(B));
  \end{lstlisting}
  kde \texttt{\{x\}pk(A)} značí zprávu \texttt{x} zašifrovanou veřejným klíčem \texttt{A}.

  %%%>==== 7.krok
  \item Nakonec přidáme bezpečnostní požadavky protokolu. V~našem případě se jedná o~utajení nonců, a~neinjektivní synchronizaci
  \begin{lstlisting}[name=NSPK]
    claim_A1(A,Secret,Na);
    claim_A2(A,Secret,Nb);
    claim_A3(A,Nisynch);
  }
  \end{lstlisting}

  %%%>==== 8.krok
  \item Obdobně přidáme i~účastníka \texttt{B} a~server \texttt{S}.
  \begin{lstlisting}[name=NSPK]
  role B
  {
    const Nb: Nonce;
    var Na: Nonce;
    recv_3(A,B,{Na,A}pk(B));
    send_4(B,S,(B,A));
    recv_5(S,B,{pk(A),A}sk(S));
    send_6(B,A,{Na,Nb}pk(A));
    recv_7(A,B,{Nb}pk(B));
    claim_B1(B,Secret,Na);
    claim_B2(B,Secret,Nb);
    claim_B3(B,Nisynch);
  }
  role S
  {
    recv_1(A,S,(A,B));
    send_2(S,A,{pk(B),B}sk(S));
    recv_4(B,S,(B,A));
    send_5(S,B,{pk(A),A}sk(S));
  }
}
  \end{lstlisting}

  %%%>==== 9.krok
  \item Nakonec přidáme nedůvěryhodného klienta, jehož soukromý klíč byl zpronevěřen útočníkem.
  \begin{lstlisting}[name=NSPK]
const C: Agent;
untrusted C;
compromised sk(C);
  \end{lstlisting}

  %%%>==== 10.krok
  \item Spuťte verifikaci protokolu(v menu \texttt{Verify->Verify protocol}), a~projděte výsledky verifikace. Nechte
  si zobrazit případný útok.
\end{enumerate}
      % Needham-Schroeder Public Key

\section{Implementace protokolu Otway Rees}
V~tomto příkladě si vyzkoušíte implementaci a~verifikaci \emph{Otway Rees}
protokolu, který používá symetrickou kryptografii.

\begin{enumerate}
  %%%>==== 1.krok
  \item Po spuštění nástroje zadávajte kód protokolu v~hlavní editovací části okna.

  %%%>==== 2.krok
  \item Na~úvod deklarujeme funkci pro symetrický klíč. Protože se jedná o~tajný
  klíč označíme funkci klíčovým slovem \texttt{secret}. Dále deklarujeme nový typ
  pro distribuovaný klíč \texttt{Key}.
  \begin{lstlisting}[name=OtwayRees]
secret k: Function;
usertype Key;
  \end{lstlisting}

  %%%>==== 3.krok
  \item Dále deklarujeme protokol klíčovým slovom \texttt{protocol} nasledovaný jeho jménem a~jmény účastníků.
  \begin{lstlisting}[name=OtwayRees]
protocol OtwayRees(A,B,S)
{
  \end{lstlisting}

  %%%>==== 4.krok
  \item Následuje definice jednotlivých uživatelů. Definice účastníka začíná klíčovým slovem \texttt{role}.
  Prvně definujeme účastníka \texttt{A} a~deklarujeme lokální proměnné pro nonce \texttt{Na} a~\texttt{M},
  a~pro obdržený klíč \texttt{Kab}.
  \begin{lstlisting}[name=OtwayRees]
  role A
  {
    const Na: Nonce;
    const M: Nonce;
    var Kab: Key;
  \end{lstlisting}

  %%%>==== 5.krok
  \item Dále definujem příchozí a~odchozí zprávy účastníka \texttt{A}.
  Příchozí a~odchozí zpráva začíná klíčovým slovem \texttt{recv}, respektive \texttt{send} následované označením zprávy.
  V~kulatých závorkách poté následuje od koho zpráva pochází, příjemce zprávy a~text zprávy.
  \begin{lstlisting}[name=OtwayRees]
    send_1(A,B,(M,A,B,{Na,M,A,B}k(A,S)));
    recv_4(B,A,(M,{Na,Kab}pk(A,S)));
  \end{lstlisting}
  kde \texttt{\{x\}pk(A,S)} značí zprávu \texttt{x} zašifrovanou sdíleným klíčem \texttt{A} a~\texttt{S}.

  %%%>==== 6.krok
  \item Nakonec přidáme bezpečnostní požadavky protokolu. V~našem případě se jedná o~utajení distribuovaného klíče \texttt{Kab}
  a~neinjektivní synchronizaci.
  \begin{lstlisting}[name=OtwayRees]
    claim_A1(A,Secret,Kab);
    claim_A2(A,Nisynch);
  }
  \end{lstlisting}

  %%%>==== 7.krok
  \item Obdobně přidáme i~účastníka \texttt{B}. Protože účastníci musí být schopni
  dešifrovat příchozí zprávy, musíme u~účasníka \texttt{B} navíc
  deklarovat proměnné \texttt{T1} a~\texttt{T2} typu \texttt{Ticket}, do kterých
  můžeme uložit zašifrovanou zprávu.
  \begin{lstlisting}[name=OtwayRees]
  role B
  {
    const Nb: Nonce;
    var M: Nonce;
    var Kab: Key;
    var T1, T2: Ticket;
    recv_1(A,B,(M,A,B,T1));
    send_2(B,S,(M,A,B,T1,{Nb,M,A,B}k(B,S)));
    recv_3(S,B,(M,T2,{Nb,Kab}k(B,S)));
    send_4(B,A,(M,T2));
    claim_B1(B,Secret,Kab);
    claim_B2(B,Nisynch);
  }
  \end{lstlisting}

  %%%>==== 8.krok
  \item Dále přidáme server \texttt{S}.
  \begin{lstlisting}[name=OtwayRees]
  role S
  {
    const Kab: Key;
    var Na, Nb, M: Nonce;
    recv_2(B,S,(M,A,B,{Na,M,A,B}k(A,S),{Nb,M,A,B}k(B,S)));
    send_3(S,B,(M,{Na,Kab}k(A,S),{Nb,Kab}k(B,S)));
  }
}
  \end{lstlisting}

  %%%>==== 9.krok
  \item Nakonec přidáme nedůvěryhodného klienta, jehož sdílený klíč byl zpronevěřen útočníkem.
  \begin{lstlisting}[name=OtwayRees]
const C,D: Agent;
untrusted C;
compromised k(C,D);
compromised k(D,C);
  \end{lstlisting}

  %%%>==== 10.krok
  \item Spuťte verifikaci protokolu(v menu \texttt{Verify->Verify protocol}), a~projděte výsledky verifikace. Nechte
  si zobrazit případný útok.
\end{enumerate}   % Otway Rees

\section{Implementace protokolu Denning-Sacco}
V~tomto příkladě si vyzkoušíte implementaci a~verifikaci protokolu \emph{Denning-Sacco}.

\begin{enumerate}
  %%%>==== 1.krok
  \item Po spuštění nástroje zadávajte kód protokolu v~hlavní editovací části okna.

  %%%>==== 2.krok
  \item Na~úvod deklarujeme funkci pro symetrický klíč. Protože se jedná o~tajný
  klíč označíme funkci klíčovým slovem \texttt{secret}. Dále deklarujeme nový typ
  pro distribuovaný klíč \texttt{Key} a~typ \texttt{Timestamp} pro časovou značku.
  \begin{lstlisting}[name=DenningSacco]
secret k: Function;
usertype Key;
usertype Timestamp;
  \end{lstlisting}

  %%%>==== 3.krok
  \item Dále deklarujeme protokol klíčovým slovom \texttt{protocol} nasledovaný jeho jménem a~jmény účastníků.
  \begin{lstlisting}[name=DenningSacco]
protocol DenningSacco(A,B,S)
{
  \end{lstlisting}

  %%%>==== 4.krok
  \item Následuje definice jednotlivých uživatelů. Definice účastníka začíná klíčovým slovem \texttt{role}.
  Prvně definujeme účastníka \texttt{A} a~deklarujeme lokální proměnné pro časovou známku \texttt{T},
  pro obdržený klíč \texttt{Kab} a~ticket pro přeposílanou zprávu \texttt{M}.
  \begin{lstlisting}[name=DenningSacco]
  role A
  {
    var T: Timestamp;
    var M: Ticket;
    var Kab: Key;
  \end{lstlisting}

  %%%>==== 5.krok
  \item Dále definujem příchozí a~odchozí zprávy účastníka \texttt{A}.
  Příchozí a~odchozí zpráva začíná klíčovým slovem \texttt{recv}, respektive \texttt{send} následované označením zprávy.
  V~kulatých závorkách poté následuje od koho zpráva pochází, příjemce zprávy a~text zprávy.
  \begin{lstlisting}[name=DenningSacco]
    send_1(A,S,(A,B));
    recv_2(S,A,{B,Kab,T,M}k(A,S));
    send_3(A,B,M);
  \end{lstlisting}
  kde \texttt{\{x\}pk(A,S)} značí zprávu \texttt{x} zašifrovanou sdíleným klíčem \texttt{A} a~\texttt{S}.

  %%%>==== 6.krok
  \item Nakonec přidáme bezpečnostní požadavky protokolu. V~našem případě se jedná o~utajení distribuovaného klíče \texttt{Kab}
  a~neinjektivní synchronizaci a~agreement.
  \begin{lstlisting}[name=DenningSacco]
    claim_A1(A,Secret,Kab);
    claim_A2(A,Nisynch);
    claim_A3(A,Niagree);
  }
  \end{lstlisting}

  %%%>==== 7.krok
  \item Obdobně přidáme i~účastníka \texttt{B}.
  \begin{lstlisting}[name=DenningSacco]
  role B
  {
    var T: Timestamp;
    var Kab: Key;
    recv_3(A,B,{Kab,A,T}k(B,S));
    claim_B1(B,Secret,Kab);
    claim_B2(B,Nisynch);
    claim_B3(B,Niagree);
  }
  \end{lstlisting}

  %%%>==== 8.krok
  \item Dále přidáme server \texttt{S}.
  \begin{lstlisting}[name=DenningSacco]
  role S
  {
    const Kab: Key;
    const T: Timestamp;
    recv_1(A,S,(A,B));
    send_2(S,A,{B,Kab,T,{Kab,A,T}k(B,S)}k(A,S));
  }
}
  \end{lstlisting}

  %%%>==== 9.krok
  \item Nakonec přidáme nedůvěryhodného klienta, jehož sdílený klíč byl zpronevěřen útočníkem.
  \begin{lstlisting}[name=DenningSacco]
const C,D: Agent;
untrusted C;
compromised k(C,D);
compromised k(D,C);
  \end{lstlisting}

  %%%>==== 10.krok
  \item Spuťte verifikaci protokolu(v menu \texttt{Verify->Verify protocol}), a~projděte výsledky verifikace. Nechte
  si zobrazit případný útok.
\end{enumerate}   % Denning-Sacco

\end{document}
