\section{Implementace protokolu Needham-Schroeder Public Key}
V~tomto příkladě si vyzkoušíte implementaci a~verifikaci \emph{Needham-Schroeder Public Key}
protokolu, který slouží pro obousměrnou autentizaci pomocí důvěryhodného serveru.

\begin{enumerate}
  %%%>==== 1.krok
  \item Po spuštění nástroje zadávajte kód protokolu v~hlavní editovací části okna.

  %%%>==== 2.krok
  \item Na~úvod deklarujeme funkce pro veřejné a~soukromé klíče. Deklarace pomocí klíčového slova \texttt{const},
  učiní funkci \texttt{pk} veřejnou. Pomocí slova \texttt{secret} označíme pak funkci \texttt{sk} jako tajnou.
  \begin{lstlisting}[name=NSPK]
const pk: Function;
secret sk: Function;
  \end{lstlisting}

  %%%>==== 3.krok
  \item Následně deklarujeme, že tyto funkce tvoří pár asymetrických klíčů.
  \begin{lstlisting}[name=NSPK]
inversekeys(pk,sk);
  \end{lstlisting}

  %%%>==== 4.krok
  \item Dále deklarujeme protokol klíčovým slovom \texttt{protocol} nasledovaný jeho jménem a~jmény účastníků.
  \begin{lstlisting}[name=NSPK]
protocol NeedhamSchroeder(A,B,S)
{
  \end{lstlisting}

  %%%>==== 5.krok
  \item Následuje definice jednotlivých uživatelů. Definice účastníka začíná klíčovým slovem \texttt{role}.
  Prvně definujeme účastníka \texttt{A} a~deklarujeme lokální proměnné pro nonce \texttt{Na}, který
  vytváří, a~pro nonce \texttt{Nb}, který obdrží od účastníka \texttt{B}.
  \begin{lstlisting}[name=NSPK]
  role A
  {
    const Na: Nonce;
    var Nb: Nonce;
  \end{lstlisting}

  %%%>==== 6.krok
  \item Dále definujem příchozí a~odchozí zprávy účastníka \texttt{A}.
  Příchozí a~odchozí zpráva začíná klíčovým slovem \texttt{recv}, respektive \texttt{send} následované označením zprávy.
  V~kulatých závorkách poté následuje od koho zpráva pochází, příjemce zprávy a~text zprávy.
  \begin{lstlisting}[name=NSPK]
    send_1(A,S,(A,B));
    recv_2(S,A,{pk(B),B}sk(S));
    send_3(A,B,{Na,A}pk(B));
    recv_6(B,A,{Na,Nb}pk(A));
    send_7(A,B,{Nb}pk(B));
  \end{lstlisting}
  kde \texttt{\{x\}pk(A)} značí zprávu \texttt{x} zašifrovanou veřejným klíčem \texttt{A}.

  %%%>==== 7.krok
  \item Nakonec přidáme bezpečnostní požadavky protokolu. V~našem případě se jedná o~utajení nonců, a~neinjektivní synchronizaci
  \begin{lstlisting}[name=NSPK]
    claim_A1(A,Secret,Na);
    claim_A2(A,Secret,Nb);
    claim_A3(A,Nisynch);
  }
  \end{lstlisting}

  %%%>==== 8.krok
  \item Obdobně přidáme i~účastníka \texttt{B} a~server \texttt{S}.
  \begin{lstlisting}[name=NSPK]
  role B
  {
    const Nb: Nonce;
    var Na: Nonce;
    recv_3(A,B,{Na,A}pk(B));
    send_4(B,S,(B,A));
    recv_5(S,B,{pk(A),A}sk(S));
    send_6(B,A,{Na,Nb}pk(A));
    recv_7(A,B,{Nb}pk(B));
    claim_B1(B,Secret,Na);
    claim_B2(B,Secret,Nb);
    claim_B3(B,Nisynch);
  }
  role S
  {
    recv_1(A,S,(A,B));
    send_2(S,A,{pk(B),B}sk(S));
    recv_4(B,S,(B,A));
    send_5(S,B,{pk(A),A}sk(S));
  }
}
  \end{lstlisting}

  %%%>==== 9.krok
  \item Nakonec přidáme nedůvěryhodného klienta, jehož soukromý klíč byl zpronevěřen útočníkem.
  \begin{lstlisting}[name=NSPK]
const C: Agent;
untrusted C;
compromised sk(C);
  \end{lstlisting}

  %%%>==== 10.krok
  \item Spuťte verifikaci protokolu(v menu \texttt{Verify->Verify protocol}), a~projděte výsledky verifikace. Nechte
  si zobrazit případný útok.
\end{enumerate}
