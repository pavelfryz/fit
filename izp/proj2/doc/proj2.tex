\documentclass[12pt,a4paper,titlepage,final]{article}

% cestina a fonty
\usepackage[czech]{babel}
\usepackage[utf8]{inputenc}
% balicky pro odkazy
\usepackage[bookmarksopen,colorlinks,plainpages=false,urlcolor=blue,unicode]{hyperref}
\usepackage{url}
% obrazky
\usepackage[dvipdf]{graphicx}
% velikost stranky
\usepackage[top=3.5cm, left=2.5cm, text={17cm, 24cm}, ignorefoot]{geometry}

\begin{document}

%%%%%%%%%%%%%%%%%%%%%%%%%%%%%%%%%%%%%%%%%%%%%%%%%%%%%%%%%%%%%%%%%%%%%%%%%%%%%%
% titulní strana

\def\author{Pavel Frýz}
\def\email{xfryzp00@stud.fit.vutbr.cz}
\def\projname{Iterační výpočty}
% !!!!!!!!!!!!!!!!!!!!!!!!!!!!!!!!!!!!!!!!!!!!!!!!!

\begin{titlepage}

% \vspace*{1cm}
\begin{figure}[!h]
  \centering
  \includegraphics[height=5cm]{img/logo.eps}
\end{figure}

\vfill

\begin{center}
\begin{Large}
Dokumentace k projektu pro předměty IZP a IUS\\
\end{Large}
\bigskip
\begin{Huge}
\projname\\
\end{Huge}
\begin{large}
projekt č. 2
\end{large}
\end{center}

\vfill

\begin{center}
\begin{Large}
\today
\end{Large}
\end{center}

\vfill

\begin{flushleft}
\begin{large}
\begin{tabular}{ll}
Autor: & \author, \url{\email} \\
 & Fakulta Informačních Technologií \\
 & Vysoké Učení Technické v~Brně \\
\end{tabular}
\end{large}
\end{flushleft}
\end{titlepage}


%%%%%%%%%%%%%%%%%%%%%%%%%%%%%%%%%%%%%%%%%%%%%%%%%%%%%%%%%%%%%%%%%%%%%%%%%%%%%%
% obsah
\pagestyle{plain}
\pagenumbering{roman}
\setcounter{page}{1}
\tableofcontents

%%%%%%%%%%%%%%%%%%%%%%%%%%%%%%%%%%%%%%%%%%%%%%%%%%%%%%%%%%%%%%%%%%%%%%%%%%%%%%
% textova zprava
\newpage
\pagestyle{plain}
\pagenumbering{arabic}
\setcounter{page}{1}

%%%%%%%%%%%%%%%%%%%%%%%%%%%%%%%%%%%%%%%%%%%%%%%%%%%%%%%%%%%%%%%%%%%%%%%%%%%%%%
\section{Úvod} \label{uvod}
%%%%%%%%%%%%%%%%%%%%%%%%%%%%%%%%%%%%%%%%%%%%%%%%%%%%%%%%%%%%%%%%%%%%%%%%%%%%%%
Tato dokumentace popisuje návrh a implementaci programu pro~výpočet vybraných
matematických funkcí. Jedná se o~$\tanh{x}$, $\log_{a}{x}$, vážený aritmetický 
a kvadratický průměr. V~kapitole~\ref{analyza} najdeme seznámení s~problémem, 
v~kapitole~\ref{reseni} najdeme řešení jednotlivých problémů, rozsah definičních
oborů a specifikaci testů. V~kapitole~\ref{popis} se nachází popis vlastní implementace
vycházející z~analýzy a návrhu řešení. V~poslední kapitole~\ref{zaver} nalezneme celkove
zhodnocení programu.


%%%%%%%%%%%%%%%%%%%%%%%%%%%%%%%%%%%%%%%%%%%%%%%%%%%%%%%%%%%%%%%%%%%%%%%%%%%%%%
\section{Analýza problému} \label{analyza}
%%%%%%%%%%%%%%%%%%%%%%%%%%%%%%%%%%%%%%%%%%%%%%%%%%%%%%%%%%%%%%%%%%%%%%%%%%%%%%
%=============================================================================
\subsection{Zadání problému}

Cílem tohoto projektu je vytvoření programu v~jazyce C pro výpočet následujích
matematických funkcí:
\begin{itemize}
 \item $y=\tanh{x}$
 \item $y=\log_{a}x$
 \item Vážený aritmetický průměr
 \item Vážený kvadratický průměr
\end{itemize}
Program musí být schopen zpracovat libovolně dlouhou posloupnost číselných
hodnot typu \texttt{double}, která budou zapsána na~standartní vstup.
Výstupní data mají být zapsána na~standartní výstup.
%=============================================================================
\subsection{Konvergence}

\paragraph*{Věta o~konvergenčním intervalu \cite{konver}\\}
Ke~každé mocninné řadě $\sum\limits_{k=0}^\infty a_{k}x^{k}$ existuje takové číslo (tzv. poloměr
konvergence) $r>0$ (připouštíme i $r = +\infty$), že pro~všechna čísla $x$, pro~něž
$|x| < r$ (tzv. interval konvergence [konvergenční interval]), je mocninná řada
absolutně konvergentní a pro~všechna čísla $x$, pro~něž $|x| > r$, je divergentní.
V~bodech $x$, pro~něž $|x|=r$, nelze obecně rozhodnout, zda mocninná řada je
konvergentní nebo divergentní. Je-li $r = 0$, pak mocninná řada je konvergentní
jen v~bodě $x = 0$. Jestliže $r = + \infty$, pak mocninná řada je konvergentní
na~množině $R$.

Vydíme, že může nastat problém pokud vstupní hodnota nepatří do~konvergenčního
intervalu. Hodnoty mimo tento interval musíme vhodně převést do~tohoto intervalu,
jinak by mohlo dojít k~zacyklení programu.

%%%%%%%%%%%%%%%%%%%%%%%%%%%%%%%%%%%%%%%%%%%%%%%%%%%%%%%%%%%%%%%%%%%%%%%%%%%%%%
\section{Řešení problému} \label{reseni}
%%%%%%%%%%%%%%%%%%%%%%%%%%%%%%%%%%%%%%%%%%%%%%%%%%%%%%%%%%%%%%%%%%%%%%%%%%%%%%
%=============================================================================
\subsection{Definiční obor funkcí}\label{rozsah}

Obor vstupních hodnot je omezen nejen definičním oborem funkcí, ale i 
použitým datovým typem \texttt{double}, který je omezen rozsahem i 
přesností na~počet platných cifer. Datový typ \texttt{double} 
také dokáže reprezentovat hodnoty $\infty, -\infty$ a NaN(Not a Number).
Každá matematická funkce, tedy musí kontrolovat jestli vstupní hodnoty spadají
do~jejich definičního oboru. S~ohledem na~použitý datový typ je definiční obor
pro~funkci logaritmus $(0;\texttt{DBL\_MAX}\rangle$, kde \texttt{DBL\_MAX} je
maximální konečná hodnota datového typu \texttt{double}, obvykle $1.79\cdot10^{308}$. Pro~hyperbolický tangens
$\left(-(\texttt{DBL\_DIG+REZERVA});\texttt{DBL\_DIG+REZERVA} \right)$, kde \texttt{DBL\_DIG} je počet
platných cifer datového typu \texttt{double}, obvykle 15. Pro~hodnoty mimo tento interval
nemá smysl funkce počítat, neboť výsledek by byl vždy 1, případně -1.

%=============================================================================
\subsection{Výpočet hyperbolického tangensu}

\begin{equation}
\label{tan1}\tanh{x}=x-\frac{1}{3}x^3+\frac{2}{12}x^5-\frac{17}{315}x^7+\ldots
\end{equation}
\begin{equation}
\label{tan2}\tanh{x}=\frac{\sinh{x}}{\cosh{x}}=
\frac{x+\frac{x^{3}}{3!}+\frac{x^{5}}{5!}+\frac{x^{7}}{7!}+\ldots}
     {1+\frac{x^{2}}{2!}+\frac{x^{4}}{4!}+\frac{x^{6}}{6!}+\ldots}
\end{equation}
\begin{equation}
\label{tan3}\tanh{x}=\frac{e^{2x}-1}{e^{2x}+1}
\end{equation}

Jedna možnost výpočtu hyperbolického tangensu je využití taylorovy řady
pro~hyperbolický tangens, vzorec~\ref{tan1}.
Druhou možnost je využít vztahu hyperbolických funkcí, vzorec~\ref{tan2}.
Třetí možností je vypočítat $e^{2x}$ a použít vzorec~\ref{tan3} vycházející z~definice
hyperbolického tangensu.

Při~bližším prozkoumání všech možností se ukázalo, že první varianta není pro~výpočet
příliš vhodná, protože vyjádření dalšího člena posloupnosti
by bylo příliš složité a neefektivní. Třetí varianta s~ohledem na~omezení datového typu
\texttt{double} není také vhodná, jelikož u~čísel blížících se nule
by nebylo dosaženo požadované přesnosti. Druhá varianta se ukázala nejvhodnější,
protože vyjádření dalšího člena posloupnosti je jednoduché a efektivní. Výhodou také je, že
uvedené řady pro~hyperbolický sinus a cosinus kovergují pro~$x\in{R}$ narozdíl 
od~řady pro~hyperbolický tangens, která konverguje pouze 
pro~$x\in{\left(-\frac{\pi}{2};\frac{\pi}{2} \right) }$.
Z~těchto důvodů jsem se rozhodl pro~druhou variantu a tu také implementoval.
%=============================================================================
\subsection{Výpočet logaritmu}

\begin{equation}
\label{log1}ln~x=2\left(\frac{x-1}{x+1}+\frac{(x-1)^{3}}{3(x+1)^{3}}+\frac{(x-1)^{5}}{5(x+1)^{5}}+\ldots \right)
\end{equation}

\begin{equation}
\label{log2}log_{a}x=\frac{ln~x}{ln~a}
\end{equation}

\begin{equation}
\label{log3}ln~xy=ln~x+ln~y
\end{equation}

\begin{equation}
\label{log4}ln~x^{n}=n\cdot ln~x
\end{equation}

Pro~výpočet $\log_{a}{x}$ můžeme použít vzorec~\ref{log2}. Výpočet přirozeného logaritmu
poté vypočteme pomocí taylorovy řady pro~přirozený logaritmus, vzorec~\ref{log1}.
Tato řada konverguje pro~$x\in\left(0;\infty \right) $.

Při~bližším prozkoumání této řady zjistíme, že řada v~uvedeném intervalu konverguje
různě rychle a že nejlépe konverguje v~blízkém okolí bodu $x=1$. Do~tohoto
okolí můžeme dostat $x$ využitím vzorců~\ref{log3} a~\ref{log4}.
%=============================================================================
\subsection{Výpočet váženého aritmetického průměru}

\begin{equation}
\label{arit}
\bar{x}=\frac{x_{1}h_{1}+x_{2}h_{2}+\ldots+x_{n}h_{n}}{h_{1}+h_{2}+\ldots+h_{n}}
=\frac{\sum\limits_{i=1}^n x_{i}h_{i}}{\sum\limits_{i=1}^n h_{i}}
\end{equation}
Výpočet aritmetického průměru provedeme pomocí vzorce~\ref{arit}. 
%=============================================================================
\subsection{Výpočet váženého kvadratického průměru}

\begin{equation}
\label{kvad}
\bar{x}_{k}=\sqrt{\frac{x^{2}_{1}h_{1}+x^{2}_{2}h_{2}+\ldots+x^{2}_{n}h_{n}}{h_{1}+h_{2}+\ldots+h_{n}}}
=\sqrt{\frac{\sum\limits_{i=1}^n x^{2}_{i}h_{i}}{\sum\limits_{i=1}^n h_{i}}}
\end{equation}
Výpočet kvadratického průměru provedeme pomocí vzorce~\ref{kvad}. 
%=============================================================================
\subsection{Specifikace testů} \label{testy}

Jelikož uživatel může zadat chybná data -- chybný parametry, nesmyslné
hodnoty (znaky, slova, apod.), je potřeba tyto stavy
otestovat a ošetřit.

\paragraph{Test 1:} Chybná parametry $\longrightarrow$ Detekce chyby.
\begin{verbatim}
./proj2 --logax text 
./proj2 --wam text 
./proj2 
./proj2 tanh 3 
./proj2 --text
\end{verbatim} 

\paragraph{Test 2:} Nesmyslné hodnoty $\longrightarrow$ Detekce chyby, výstup = nan.
\begin{verbatim}
./proj2 --tanh 14 <<<"0 12text2 0 text 0 0 12text"
\end{verbatim} 
\vspace{1em}\begin{tabular}{l} % ll = 2 sloupce zarovnane: left,left
očekávaný výstup \\
\hline
0.0000000000e+00 \\
nan \\
0.0000000000e+00 \\
nan\\
0.0000000000e+00 \\
0.0000000000e+00 \\
nan \\
\end{tabular}

\paragraph{Test 3:} Správnost výpočtu $\tanh{x}$ $\longrightarrow$ Předpokládaná správná hodnota.
\begin{verbatim}
./proj2 --tanh 14
\end{verbatim}
\vspace{1em}\begin{tabular}{ll} % ll = 2 sloupce zarovnane: left,left
vstup & očekávaný výstup \\
\hline
\verb|12.4536| & 9.9999999997e-01 \\
\verb|-0.234| & -2.2982054821e-01 \\
\verb|0.234| &   2.2982054821e-01 \\
\verb|23e-300| & 2.3000000000e-299\\
\verb|2e303| & 1.0000000000e+00 \\
\verb|-inf| & -1.0000000000e+00 \\
\verb|nan| & nan \\
\verb|0| & 0.0000000000e+00 \\
\end{tabular}

\paragraph{Test 4:} Správnost výpočtu $\log_{a}{x}$ $\longrightarrow$ Předpokládaná správná hodnota.
\begin{verbatim}
./proj2 --logax 14 1.423
\end{verbatim}
\vspace{1em}\begin{tabular}{ll} % ll = 2 sloupce zarovnane: left,left
vstup & očekávaný výstup \\
\hline
\verb|13.4536| & 7.3681619227e+00 \\
\verb|-3532.234| & nan \\
\verb|23543254.24| & 4.8117693941e+01 \\
\verb|23e-300| & -1.9492736329e+03\\
\verb|2e303| & 1.9797084155e+03 \\
\verb|inf| & inf \\
\verb|nan| & nan \\
\verb|0| & -inf \\
\end{tabular}

\paragraph{Test 5:} Správnost výpočtu aritmetického průměru $\longrightarrow$ Předpokládaná správná hodnota.
\begin{verbatim}
./proj2 --wam <<<"1 12 2 2 12.321 1 1803.32"
\end{verbatim} 
\vspace{1em}\begin{tabular}{l} % ll = 2 sloupce zarovnane: left,left
očekávaný výstup \\
\hline
1.0000000000e+00 \\
1.1428571429e+00 \\
1.8880666667e+00 \\
nan\\
\end{tabular}

\paragraph{Test 6:} Správnost výpočtu kvadratického průměru $\longrightarrow$ Předpokládaná správná hodnota.
\begin{verbatim}
./proj2 --wqm <<<"1 12 2 2 12.321 1 1803.32"
\end{verbatim} 
\vspace{1em}\begin{tabular}{l} % ll = 2 sloupce zarovnane: left,left
očekávaný výstup \\
\hline
1.0000000000e+00 \\
1.1952286093e+00 \\
3.3843467218e+00 \\
nan\\
\end{tabular}

%%%%%%%%%%%%%%%%%%%%%%%%%%%%%%%%%%%%%%%%%%%%%%%%%%%%%%%%%%%%%%%%%%%%%%%%%%%%%%
\section{Popis řešení} \label{popis}
%%%%%%%%%%%%%%%%%%%%%%%%%%%%%%%%%%%%%%%%%%%%%%%%%%%%%%%%%%%%%%%%%%%%%%%%%%%%%%

Při~implementaci jsem vycházel ze~závěrů popsaných v~předchozích kapitolách.

%=============================================================================
\subsection{Ovládání programu}

Program funguje jako konzolová aplikace, má tedy pouze textové ovládání. Při~spouštění
programu s~parametr \texttt{-h} vypíše obrazovku s~nápovědou.
Další možnost spuštění je s~parametrem \texttt{--wam} nebo \texttt{--wqm},
při~tomto spuštění se počítá vážený aritmetický průměr nebo vážený kvadratický
průměr. Při spuštění s~parametrem \texttt{--tanh} nebo \texttt{--logax} očekavá
program ještě druhý parametr, a to přesnost na~počet platných cifer. A v~případě
výpočtu logaritmu očekává ještě jeden parametr reprezentující základ logaritmu. 

Po~spuštění s~těmito parametry program očekává na~standardním vstupu 
hodnoty typu \texttt{double} oddělené posloupností bílých znaků.
Každá hodnota je předána zvolené funkci a výsledek operace je vytisknut
na standartní výstup v~zadaném formátu.

Výhodou tohoto ovládání je, že program může být použít ve~skriptech
a~jím produkovaný výsledek může být použit jiným
programem pro~další výpočet.

%=============================================================================
\subsection{Volba datových typů}

Pro~uložení hodnot výsledku jsem zvolil datový typ \texttt{double}.
Pro~uložení hodnot pro~výpočet vážených průměrů slouží struktura \texttt{TAverage},
která obsahuje čtyři položky typu \texttt{double} pro~sou\-čet hodnot, součet vah, aktualní hodnotu
a aktuální váhu.

%=============================================================================
\subsection{Vlastní implementace}

Parametry příkazové řádky zpracovává funkce \texttt{getParams}, která je
spouštěna jako první ve~funk\-ci \texttt{main}. Načtené parametry se poté
předají funkci \texttt{calculate}, která v~cyklu pomocí funkce \texttt{scanf}
načítá vstupní data a pomocí funkce \texttt{testInput} detekuje případné chyby.
Poté je podle parametrů zavolaná funkce pro výpočet zvolené funkce. Výsledek
je poté vytisknut pomocí funkce \texttt{printf} ve zvoleném formátě.

Pro~výpočet váženého kvadratického průměru slouží funkce \texttt{wam}, která má jediný
parametr strukturu \texttt{TAverage}, která obsahuje součet hodnot, součet vah a aktualní hodnotu a váhu.
Z~těchto hodnot vypočte aritmetický průměr a příslušně změní součet hodnot a vah.
Velice podobně funguje funkce \texttt{wqm} pro~výpočet kvadratického aritmetického průměru.
Pro~výpočet hyperbolického tangensu slouží funkce \texttt{mytanh}.
Výpočet $\log_{a}x$ zajišťuje funkce \texttt{logax}. Funkce prvně vypočítá 
pomocí funkce \texttt{ln} přirozený logaritmus pro~základ a hodnotu. A tyto hodnoty
poté podělí podle vzorce~\ref{log1}. 

%%%%%%%%%%%%%%%%%%%%%%%%%%%%%%%%%%%%%%%%%%%%%%%%%%%%%%%%%%%%%%%%%%%%%%%%%%%%%%
\section{Závěr} \label{zaver}
%%%%%%%%%%%%%%%%%%%%%%%%%%%%%%%%%%%%%%%%%%%%%%%%%%%%%%%%%%%%%%%%%%%%%%%%%%%%%%
Výsledný algoritmus počítá s~poměrně velkou přesností. Zvolené algoritmy
počítají co nejefektivněji. Program skončí i pro mezní hodnoty
v~rozumném čase.

Navržené řešení je přenositelné na všechny platformy, které podporují
datový typ \texttt{double}. Při~přenosu na~platformu, která tento typ nepodporuje,
by bylo nutné upravit datové typy.

Program striktně dodržuje formát vstupních a výstupních dat, díko tomu je možné
využití ve~skriptech nebo spolupráce s~jinými programi.


%%%%%%%%%%%%%%%%%%%%%%%%%%%%%%%%%%%%%%%%%%%%%%%%%%%%%%%%%%%%%%%%%%%%%%%%%%%%%%
% seznam citované literatury: každá položka je definována příkazem
% \bibitem{xyz}, kde xyz je identifikátor citace (v textu použij: \cite{xyz})
\begin{thebibliography}{1}

% jedna citace:
\bibitem{konver}
BARTSCH, Hans-Jochen: \emph{Matematické vzorce}. Praha: Mladá fronta, 1996, ISBN 80-204-0607-7.


\end{thebibliography}
%%%%%%%%%%%%%%%%%%%%%%%%%%%%%%%%%%%%%%%%%%%%%%%%%%%%%%%%%%%%%%%%%%%%%%%%%%%%%%
% přílohy
\appendix

\section{Metriky kódu} \label{metriky}
\paragraph{Počet souborů:} 1 soubor
\paragraph{Počet řádků zdrojového textu:} 687 řádků
\paragraph{Velikost statických dat:} 7597B
\paragraph{Velikost spustitelného souboru:} 12164B (systém Linux, 64 bitová
architektura, při překladu bez ladicích informací)


\end{document}
