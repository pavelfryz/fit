\documentclass[10pt,a4paper,twocolumn]{article}
\usepackage[czech]{babel}
\usepackage[utf8]{inputenc}
\usepackage[T1]{fontenc}
\usepackage[top=2.5cm, left=1.5cm, text={18cm, 25cm}]{geometry}

\title{Typografická etiketa}
\author{Vysázel: Pavel Frýz\\
      xfryzp00@stud.fit.vutbr.cz}
\date{}

\begin{document}
\maketitle

\section{Hladká sazba}

Hladká sazba používá jeden stupeň, druh a řez písma a je sázena
na stanovenou šířku odstavce. Je složena z~od\-stav\-ců, obvykle
začínajících zarážkou, nejde-li o~první odstavec za nadpisem.
Mohou ale být sázeny i bez zarážky\,--\,roz\-ho\-du\-jí\-cí je celková
grafická úprava. Odstavec končí vý\-cho\-do\-vou řádkou. Každá věta
začíná velkým písmenem, nesmí začínat číslicí.

Zvýraznění barvou, podtržením, ani změnou písma se v~odstavcích
nepoužívá. Hladká sazba je určena především pro delší texty,
jako je beletrie. Porušení konzistence sazby působí v~textu
rušivě a unavuje čtenářův zrak.

\section{Smíšená sazba} \label{smisenasazba}

Smíšená sazba má volnější pravidla. Klasická hladká sazba se
doplňuje o~další řezy písma pro zvýraznění důležitých
pojmů. Existuje „pravidlo“:

\begin{quotation}
Čím více \texttt{druhů,} \textbf{\emph{řezů,}} {\small velikostí,} barev
písma \textsf{a jiných efektů} použi{\large jeme,} tím
\emph{profesionálněji} {\Large bude} dokument vypadat. Čtenář
{\tiny bude} \textbf{nadšen!}
\end{quotation}


\textsc{Tímto pravidlem se nikdy nesmíme řídit.} Příliš časté
zvýrazňování textových elementů a změny ve\-li\-kos\-ti písma jsou
známkou amatérismu autora a působí velmi rušivě. Dobře
navržený dokument nemá obsahovat více než 4 řezy či druhy
písma. Dobře navržený dokument je decentní, ne chaotický.

Důležitým znakem správně vysázeného dokumentu je
konzistence\,--\,například \textbf{tučný řez} písma vyhradíme pou\-ze
pro klíčová slova, \emph{skloněný řez} pou\-ze pro doposud ne\-zná\-mé pojmy a
nebudeme to míchat. Skloněný řez ne\-pů\-so\-bí tak rušivě a používá se
častěji. V~\LaTeX u jej sázíme raději příkazem \verb|\emph{text}| než
\verb|\textit{text}|.

Smíšená sazba se nejčastěji používá pro sazbu vě\-dec\-kých
článků a technických zpráv. U~delších dokumentů vědeckého či
technického charakteru je zvykem vysvětlit význam různých typů
zvýraznění v~úvodní kapitole.


\section{Další rady:} \label{dalsirady}
\begin{itemize}
\item Nadpis nesmí končit dvojtečkou a nesmí obsahovat odkazy na
obrázky, citace, poznámky pod čarou,\,\dots

\item Nadpisy, číslování a odkazy na číslované elementy mu\-sí
být sázeny příkazy k~tomu určenými. Maximálně využíváme
možností \LaTeX u a zvolené třídy do\-ku\-men\-tu.

\item Výčet ani obrázek nesmí začínat hned pod nadpisem a nesmí
tvořit celou kapitolu.

\item Poznámky pod čarou\footnote{Příliš mnoho poznámek pod
čarou čtenáře zbytečně rozptyluje.} používejte opravdu
střídmě. (Šetřete i s~textem v~závorkách.)

\item Nepoužívejte velké množství malých obrázků. Zvažte, zda
je nelze seskupit.

\item Bezchybným pravopisem a sazbou dáváme najevo úctu ke
čtenáři. Odbytý text s~chybami bude čtenář právem považovat za
nedůvěryhodný.
\end{itemize}

\section{České odlišnosti}

Česká sazba se oproti okolnímu světu v~některých aspektech
mírně liší. Jednou z~odlišností je sazba uvozovek. Uvozovky se
v~češtině používají převážně pro zobrazení přímé řeči,
zvýraznění přezdívek a ironie. V~češtině se používají
uvozovky typu „9966“ místo anglických “uvozovek” nebo dvojitých
"uvozovek". Lze je sázet připravenými příkazy nebo při použití
UTF-8 kódování i přímo. Obě možnosti mají své výhody i úskalí.

Ve smíšené sazbě se řez uvozovek řídí řezem prvního
uvozovaného slova. Pokud je uvozována celá věta, sází se koncová
tečka před uvozovku, pokud se uvozuje slovo nebo část věty, sází
se tečka za uvozovku.

Druhou odlišností je pravidlo pro sázení konců řádků. V~české
sazbě by řádek neměl končit osamocenou jednopísmennou
předložkou nebo spojkou. Spojkou „a“ končit může pouze při sazbě
do šířky 25 liter. Abychom \LaTeX u zabránili v~sázení osamocených
předložek, spojujeme je s~následujícím slovem nezlomitelnou
mezerou. Tu sázíme pomocí znaku \verb|~| (vlnka, tilda). Pro systematické
doplnění vlnek slouží volně šiřitelný program \emph{vlna} od pana
Olšáka\footnote{Viz \texttt{ftp://math.feld.cvut.cz/pub/olsak/vlna/}}.

Principiálně lepší řešení nabízí balík \emph{encxvlna}, od pánů
Olšáka a Wagnera\footnote{Viz \texttt{http://tug.ctan.org/pkg/encxvlna}}. Pro
jeho použití je ovšem potřeba speciální konfigurace \LaTeX u.


\section{Závěr}

Jistě jste postřehli, že tento dokument obsahuje schválně několik
typografických prohřešků. Sekce \ref{smisenasazba} a
\ref{dalsirady} obsahují typografické chyby. V~kontextu celého textu
je jistě snadno najdete. Je dobré znát možnosti \LaTeX u, ale je
také nutné vědět, kdy je nepoužít.

\end{document}