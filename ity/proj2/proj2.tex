\documentclass[11pt,a4paper,twocolumn]{article}
\usepackage[czech]{babel}
\usepackage[utf8]{inputenc}
\usepackage[T1]{fontenc}
\usepackage[top=2.5cm, left=1.5cm, text={18cm, 25cm}]{geometry}

\usepackage{mdwlist}
\usepackage{amsthm}
\usepackage{amsmath}
\usepackage{amssymb}

\newtheorem{definice}{Definice}[section]
\newtheorem{veta}{Věta}
\newtheorem{algoritmus}[definice]{Algoritmus}

\begin{document}

\begin{titlepage}
  \begin{center}
    \LARGE \textsc{Vysoké učení technické v~Brně             \\
    \Large         Fakulta informačních technologií          \\
    \vspace{\stretch{0.382}}}
    \LARGE         Typografie a~publikování\,--\,2. projekt  \\
    \Huge          Sazba dokumentů s~matematickými výrazy    \\
    \vspace{\stretch{0.618}}
  \end{center}
  \Large           Pavel Frýz \hfill \today
\end{titlepage}

\section{Úvod}
  Tato úloha je zaměřena na sazbu titulní strany a~textů, které obsahují matematické
  vzorce, rovnice (\ref{rovnice1}), prostředí (například definice \ref{bezkontextgram}
  na straně \pageref{bezkontextgram}).

  Na titulní straně je využito sázení nadpisu podle optického středu s~využitím
  \emph{zlatého řezu}. Tento postup byl probírán na přednášce. Pro sazbu matematických
  elementů byly využity balíky \AmS-\LaTeX u.

\section{Plynulý matematický text}
  Zásady pro sazbu matematiky v~plynulém textu odpovídají zásadám pro smíšenou sazbu.

  Pro množinu $M$ označuje $\mathrm{card}(M)$ kardinalitu $M$. Pro množinu $M$
  reprezentuje $M^*$ volný monoid generovaný množinou $M$ s~operací konkatenace. Prvek
  identity ve volném monoidu $M^*$ značíme symbolem $\varepsilon$. Nechť $M^+ = M^* -
  \{ \varepsilon \}$. Algebraicky je tedy $M^+$ volná pologrupa generovaná množinou $M$
  s~operací konkatenace. Konečnou neprázdnou množinu $M$ nazvěme \emph{abeceda}. Pro
  $w \in M^*$ označuje $|w|$ délku řetězce $w$. Pro $W \subseteq M$ označuje
  $\mathrm{occur}(w,W)$ počet výskytů symbolů z~$W$ v~řetězci $w$ a~$\mathrm{sym}(w,i)$
  určuje $i$-tý symbol řetězce $w$; například $\mathrm{sym}(abcd,3) = c$.

\section{Sazba definic a~vět}
  Pro sazbu definic a~vět slouží balík \texttt{amsthm}.

  \begin{definice}\label{bezkontextgram}
    \emph{Bezkontextová gramatika} je čtveřice $G=(V,T,P,S)$, kde
    \begin{description*}
      \item[$V$] je totální abeceda,
      \item[$T\subseteq V$] je abeceda terminálů,
      \item[$S\in(V-T)$] je startující symbol,
      \item[$P$] je konečná množina \emph{pravidel} tvaru $q\colon A\rightarrow \alpha$,
        kde $A\in(V-T)$, $\alpha\in V^*$ a~$q$ je návěští tohoto pravidla.
    \end{description*}
    Nechť $N = V - T$ značí abecedu neterminálů. Pokud $q \colon A \rightarrow \alpha
    \in P$, $\gamma, \delta \in V^*$, $G$ provádí derivační krok z~$\gamma A \delta$
    do $\gamma \alpha \delta$ podle pravidla $q \colon A \rightarrow \alpha$, symbolicky
    píšeme $\gamma A \delta \Rightarrow \gamma \alpha \delta \; [q \colon A \rightarrow
    \alpha]$ nebo zjednodušeně $\gamma A \delta \Rightarrow \gamma \alpha \delta$.
    Standardním způsobem definujeme $\Rightarrow^n$, kde $n \geq 0$. Dále definujeme
    tranzitivní uzávěr $\Rightarrow^+$ a~tranzitivně-reflexivní uzávěr $\Rightarrow^*$.
  \end{definice}

  Algoritmus můžeme uvádět textově, podobně jako definice, nebo lze použít pseudokódu
  vysázeného ve vhodném prostředí (například \texttt{algorithm2e}).

  \begin{algoritmus}
    \emph{Ověření bezkontextovosti gramatiky}. Mějme gramatiku $G=(N,T,P,S)$.
    \begin{enumerate}
      \item \label{prvnikrok}Pro každé pravidlo $p \in P$ proveď test, zda $p$ na levé
        straně obsahuje právě jeden symbol z~$N$.
      \item Pokud všechna pravidla splňují podmínku z~kroku \ref{prvnikrok}, tak je
        gramatika $G$ bezkontextová.
    \end{enumerate}
  \end{algoritmus}

  \begin{definice}
    \emph{Jazyk} definovaný gramatikou $G$ definujeme jako $L(G) = \{ w \in T^* \; | \;
    S \Rightarrow^* w \}$.
  \end{definice}

\subsection{Podsekce obsahující větu}
  Věty a~definice mohou mít vzájemně nezávislé číslování. Důkaz se obvykle uvádí hned
  za větou.

  \begin{definice}
    Nechť $L$ je libovolný jazyk. $L$ je \emph{bezkontextový jazyk}, když a~jen když
    $L = L(G)$, kde $G$ je libovolná bezkontextová gramatika.
  \end{definice}

  \begin{definice}
    Množinu $\mathcal{L}_{CF} = \{L|L$ je bezkontextový jazyk$\}$ nazýváme \emph{třídou
    bezkontextových jazyků}.
  \end{definice}
  \begin{veta}\label{prvniveta}
    Nechť $L_{abc}=\{a^nb^nc^n|n\geq0\}$. Platí, že $L_{abc}\notin \mathcal{L}_{CF}$.
  \end{veta}
  \begin{proof}
    Důkaz se provede pomocí Pumping lemma pro bezkontextové jazyky a~je zřejmý, což
    implikuje pravdivost věty \ref{prvniveta}.
  \end{proof}

\section{Rovnice a~odkazy}
  Složitější matematické formulace sázíme mimo plynulý text. Lze umístit několik výrazů
  na jeden řádek, ale pak je třeba tyto vhodně oddělit, například příkazem \verb|\quad|.

  \[
    \sqrt[a^8]{^3_4b^2_1}          \quad
    \mathbb{N} = \{1,2,3,\ldots\}  \quad
    x^{y^y} \neq x^{yy}            \quad
    z_{i_j} \not \equiv z_{ij}
  \]

  V~rovnici (\ref{rovnice1}) jsou využity tři typy závorek s~různou explicitně definovanou
  velikostí.

  \begin{eqnarray}
    \label{rovnice1}
    x & = & -\bigg(\Big\{\big[a \ast b \big]^c - d \Big\} + 1 \bigg)              \\
    s & = & \sqrt{\frac{1}{n}\sum^s_{i=1} p_i (x_i-x)^2} \nonumber
  \end{eqnarray}

  V~této větě vidíme, jak vypadá implicitní vysázení limity $\lim_{n \rightarrow \infty}
  f(n)$ v~normálním odstavci textu. Podobně je to i s~dalšími symboly jako $\sum^n_1$ či
  $\bigcup_{A \in \mathcal{B}}$. V~případě vzorce $\lim\limits_{x \rightarrow 0}
  \frac{\sin x}{x} = 1$ jsme si vynutili méně úspornou sazbu příkazem \verb|\limits|.

  \begin{eqnarray}
    \int^b_a f(x)\,\mathrm{d}x & = & -\int\limits^a_b f(x)\,\mathrm{d}x                 \\
    \overline{\overline{A}\wedge \overline{B}} & = &  \overline{\overline{A\vee B}}
  \end{eqnarray}

\section{Složené zlomky}
  Při sázení složených zlomků dochází ke zmenšování použitého písma v~čitateli
  a~jmenovateli. Toto chování není vždy žádoucí, protože některé zlomky potom mohou být
  obtížně čitelné.

  V~těchto případech je možné nastavit standardní stupeň písma v~podvýrazech ručně pomocí
  příkazu \verb|\displaystyle| u~vysázených vzorců nebo pomocí \verb|\textstyle| u~vzorců,
  které jsou součástí textu. Srovnejte:
  \[
    \frac{\displaystyle\frac{(a+b)^2}{x+y}-\frac{x-y}{\displaystyle\frac{a}{b}}}
      {1-\displaystyle\frac{a+b}{a-b}}
    \quad
    \frac{\textstyle\frac{(a+b)^2}{x+y}-\frac{x-y}{\frac{a}{b}}}
      {1-\textstyle\frac{a+b}{a-b}}
  \]

  Tento postup lze použít nejen u~zlomků.
  \[
    \prod_{i=0}^{m-1} (n-i) = \overbrace{n(n-1)(n-2) \ldots (n-m+1)}^{\displaystyle m
    \text{ je počet činitelů}}
  \]

\section{Matice}
  Pro sázení matic se velmi často používá prostředí \texttt{array} a~závorky (\verb|\left|,
  \verb|\right|). Tyto příkazy vždy tvoří pár a~nelze je použít samostatně.

  \[
    \left(
      \begin{array}{c c}
        a+b             & a-b                       \\
        \widetilde{c+d} & \tilde{b}                 \\
        \vec{a}         & \underleftrightarrow{AC}  \\
        \xi             & \aleph
      \end{array}
    \right)
  \]
  \[
    \mathbf{A} =
    \left\|
      \begin{array}{c c c c}
        a_{11} & a_{12} & \ldots & a_{1n}  \\
        a_{21} & a_{22} & \ldots & a_{2n}  \\
        \vdots & \vdots & \ddots & \vdots  \\
        a_{m1} & a_{m2} & \ldots & a_{mn}
      \end{array}
    \right\|
  \]
  \[
    \left|
      \begin{array}{c c}
        h & i  \\
        w & x
      \end{array}
    \right| = hx - iw
  \]

  Prostředí \texttt{array} lze úspěšně využít i jinde.

  \[
    \binom{n}{k} =
    \left\{
      \begin{array}{l l}
        0                   & \quad \text{pro } k<0 \text{ nebo } k>n  \\
        \frac{n!}{k!(n-k)!} & \quad \text{pro } 0 \leq k \leq n
      \end{array}
    \right.
  \]

\section{Závěrem}
  V~případě, že budete potřebovat vyjádřit ma\-te\-ma\-tickou konstrukci nebo symbol a~nebude se
  Vám dařit jej nalézt v~samotném \LaTeX u, doporučuji prostudovat možnosti balíku maker
  \AmS-\LaTeX. Analogická poučka platí obecně pro jakoukoli konstrukci v~\TeX u.
\end{document}
