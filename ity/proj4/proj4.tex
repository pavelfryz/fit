\documentclass[12pt,a4paper]{article}
\usepackage[czech]{babel}
\usepackage[utf8]{inputenc}
\usepackage[T1]{fontenc}
\usepackage[top=3cm, left=2cm, text={17cm, 24cm}]{geometry}
\usepackage{url}

\begin{document}
\pagestyle{empty}
\LaTeX je systém pro sazbu textu, který obsahuje sadu nástrojů
pro produkci technických a~vědeckých dokumentů. Poprvé byl
\LaTeX představen Leslie Lamportem v~roce 1985, v~současnosti je
spravován a~vyvíjen tvůrčí skupinou \LaTeX3\cite{latexproj}.
Systém je založen na nástroji \TeX, který vytvořil profesor
Donald E. Knuth. Podle tvůrce nápad na vytvoření \TeX u vznikl
1.\,února 1977, kdy Knuth náhodouviděl výstup ze sázecího stroje
s~vysokým rozlišením\cite{digtyp}.
Zdrojové soubory mají obvykle příponu .tex a~mohou být vytvořeny v~libovolném
textovém editoru. Překladem vstupních souborů vzniká soubor s~příponou .dvi(device independent),
výstup by měl být nezávislý na výstupním zařízení, kterým může být například tiskárna nebo obrazovka
počítače\cite{vsbbenes}.
Zdrojový soubor se skládá z~preambule a textové části. V~povinném úvodním
příkazu se musí určit třída dokumentu, která určuje styl sazby.
Textová část je ohraničena příkazy \verb|\begin{document}| a~\verb|\end{document}|\cite{rybicka}.

Při sazbě textu musíme dodržovat určitá typografická pravidla\cite{typomil}. Musíme si taky
rozmyslet jestli je \LaTeX vhodné použít na sazbu daného dokumentu\cite{wwfu}. Pro výuku
\LaTeX u vznikly jak knihy, tak také články na internetu a~v~odborných publikacích\cite{linuxj,latesec}
a~bakalalářských prací\cite{simek}.

Pro sazbu bibliografických citací lze použít nástroj Bib\TeX a~styl czplain.bst\cite{pysny}, pomocí
kterého byl vysázen i~tento dokument.
%\cite{cj97}% \cite{repr}

\bibliographystyle{czplain}
\renewcommand{\refname}{Literatura}
\bibliography{literatura}

\end{document}